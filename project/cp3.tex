% use the answers clause to get answers to print; otherwise leave it out.
\documentclass{article}
%\documentclass[12pts, answers]{exam}
%\documentclass[12pts]{exam}
\RequirePackage{amssymb, amsfonts, amsmath, latexsym, verbatim, xspace, setspace}

% By default LaTeX uses large margins.  This doesn't work well on exams; problems
% end up in the "middle" of the page, reducing the amount of space for students
% to work on them.
\usepackage[margin=1in]{geometry}
\usepackage{enumerate}
\usepackage{color}
\usepackage[hidelinks]{hyperref}

% Here's where you edit the Class, Exam, Date, etc.
\newcommand{\class}{NPRE 560}
\newcommand{\term}{Spring 2019}
\newcommand{\assignment}{Computational Project 3: Option A}
\newcommand{\duedate}{2019.05.03}
%\newcommand{\timelimit}{50 Minutes}

\newcommand{\nth}{n\ensuremath{^{\text{th}}} }
\newcommand{\ve}[1]{\ensuremath{\mathbf{#1}}}
\newcommand{\Macro}{\ensuremath{\Sigma}}
\newcommand{\vOmega}{\ensuremath{\hat{\Omega}}}

% For an exam, single spacing is most appropriate
\singlespacing
% \onehalfspacing
% \doublespacing

% For an exam, we generally want to turn off paragraph indentation
\parindent 0ex

%\unframedsolutions
\usepackage{bibentry}
\begin{document} 

% These commands set up the running header on the top of the exam pages
%\pagestyle{head}
%\firstpageheader{}{}{}
%\runningheader{\class}{\assignment\ - Page \thepage\ of \numpages}{Due \duedate}
%\runningheadrule

\class \hfill \term \\
\assignment \hfill Due \duedate\\
%\begin{flushright}
%\begin{tabular}{p{5in} r l}
%\end{tabular}
%\end{flushright}
\rule[1ex]{\textwidth}{.1pt}

%%%%%%%%%%%%%%%%%%%%%%%%%%%%%%%%%%%%%%%%%%%%%%%%%%%%%%%%%%%%%%%%%%%%%%%%%%%%%%%%%%%%%
%%%%%%%%%%%%%%%%%%%%%%%%%%%%%%%%%%%%%%%%%%%%%%%%%%%%%%%%%%%%%%%%%%%%%%%%%%%%%%%%%%%%%

In \class, a large part of your grade is earned with computational projects. 
These are intended to build your technical writing skills, tie together the 
lessons of the course, and hone your existing computational and problem solving 
skills. This assignment must be submitted via GitHub Classroom. \textbf{To claim your 
repository, go to this link: \url{https://classroom.github.com/a/kSkYVsZn}}.

Please submit:
\begin{itemize} 
        \item The software developed.
        \item A report describing your solutions.
        \item A 10 minute presentation, to be given on Monday, May 3rd, 7-10pm.
\end{itemize}

\section{Problem Option A: Incorporate Physical Feedbacks}

Based on your computer program from CP2, solve the point reactor kinetics equations for an 
arbitrary number of delayed neutron precursor groups. This time, in addition to 
Doppler feedback, incorporate coupled equations that capture the physics and 
time dependence of at least two additional types of reactor feedback 
that might arise in a transient. 

For example, the following sources of reactivity feedback could be incorporated:

\begin{itemize}
        \item Axial, radial, or volumetric fuel expansion due to temperature change
        \item Moderator or coolant density 
        \item Coolant displacement
\end{itemize}

You must provide, explain, and defend the mathematical models that you select for 
your implementation of these feedbacks. You are responsible for providing 
and citing the sources of necessary supportive data.

Demonstrate the impact of feedback in the scenarios you have simulated in 
previous computational projects:

\begin{itemize}
        \item Control Rod Ejection
        \item Reactivity Ramp
        \item Sinusoidal Reactivity 
\end{itemize}



\section{Problem Option B: Incorporate Spatial Effects}
Using any method discuss in the book, in class, or in the journal articles, 
implement a method for capturing spatial kinetics. Solve for the behavior in a 
reactor due to an asymmetric reactivity ramp \textbf{or} control rod ejection. 
compare that solution to the corresponding solution incorporating no spatial 
asymmetry (the point kinetics approach from CP1).

\section{Software Expectations}

The program must solve this system \textbf{numerically}, and the 
implementation should be clearly described and the results should be discussed 
\textbf{analytically} in the report.  The analytical discussion should 
follow relevant discussion in your book.  The main program should 
call functions/subroutines to do the following:

\begin{enumerate}
        \item Read input file with the input data
        \item Perform all problems using a stable solution method and 
                timestep sizes.
        \item Write results to an output file.  The output file should also 
                contain the parameters read from the input file.
        \item Read the output file and use it to plot the results.
\end{enumerate}




\section{Report Content}
Submit a brief report with your results. 
See later section on the report format.

\subsection{Part 1: Theory}
Introduce the theory behind the simulations. Include the following information 
about your solution mathematics:

\begin{enumerate}
        \item Show data and sources of data that are essential to your solution {\color{red}\textbf{[-10\%]}}.
        \item Show differential equations you are solving  {\color{red}\textbf{[-10\%]}}.
        \item Discuss analytic solution of the differential equations  {\color{red}\textbf{[-10\%]}}.
        \item Show complete derivation of numerical solution {\color{red}\textbf{[-10\%]}}.
\end{enumerate}

\subsection{Part 2: Results}

Report the results of your calculations.

\begin{itemize}
        \item Solve the problems listed plotting both power and reactivity as a function of time.  
                {\color{red}\textbf{[-20\%]}}.
        \item Create plots that demonstrate the importance of physical 
                feedbacks (option A) or spatial effects (option B).
                {\color{red}\textbf{[-20\%]}}.
\end{itemize}

\section{Report Formatting}
\subsection{General}
Please write a comprehensive, self-contained report.
                \begin{itemize}
                        \item It must be computer generated,not hand written 
                                {\color{red}\textbf{[-5\%]}}.
                        \item PDF must be submitted via classroom.github.com by 
                                the due date  {\color{red}\textbf{[-5\%]}}.
                        \item Submit computer program via github classroom 
                                 {\color{red}\textbf{[-20\%]}} including sample input  {\color{red}\textbf{[-20\%]}} 
                                and sample output  {\color{red}\textbf{[-20\%]}} by the deadline.
                        \item    Input/output can be written in any format that is convenient 
                                for you.  Input and output file should contain sufficient 
                                information such that 3rd person could understand what it 
                                contains  {\color{red}\textbf{[-5\%]}}.
                \end{itemize}

                \subsection{Content Formatting}
        \begin{itemize}
        \item    Report should be self-contained, do not repeat assignment text 
                verbatim, do not copy/paste the assignment itself  {\color{red}\textbf{[-5\%]}}.
        \item    Do not submit results as raw ``column of numbers'' data  {\color{red}\textbf{[-5\%]}}.
        \item    Do not include your source code in the report  {\color{red}\textbf{[-5\%]}}.
        \item    Snippets (small parts) of the source code are OK, if relevant. 
                Consider using the \LaTeX \texttt{minted} package for syntax highlighting, if 
                you're using \LaTeX.
        \item    Do not include commands typed in the prompt (Matlab, shell, compiler, etc) 
                 {\color{red}\textbf{[-5\%]}}.
        \item    Do not include extra plots/figures  {\color{red}\textbf{[-5\%]}}.
        \item    Additional figures that support the requested results are OK.
        \item   Report length should be less than 15 pages, if you exceed 15 pages, 
                you are probably doing something wrong, for example:
                \begin{itemize}
                        \item  1-3 pages for problem description, equations, analytic solution, derivations
                        \item  1-3 pages for problem 1 (problem description, results, discussion)
                        \item  1-3 pages for problem 2 (problem description, results, discussion)
                \end{itemize}
        \item    Obviously wrong solution  {\color{red}\textbf{[-5\%]}}.
        \item    Include well-formatted references  {\color{red}\textbf{[-5\%]}} 
\end{itemize}

\subsection{Formatting}
\begin{itemize}
        \item Cover page with your name, assignment title/number, course number, date  {\color{red}\textbf{[-5\%]}}.
        \item     Include page numbers, except on the title/cover page  {\color{red}\textbf{[-5\%]}}.
        \item Report body has to start on page 1  {\color{red}\textbf{[-5\%]}}.
        \item    Use portrait orientation  {\color{red}\textbf{[-5\%]}}.
        \item    Landscape for a single page with large table/figure is OK.
        \item    Separate and clearly label each problem/exercise, so that it is easy to 
                see where one ends and the other begins.  For multiple part of the exercise, 
                separate each sub-exercise so it is easy to find  {\color{red}\textbf{[-5\%]}}.
        \item Plots, figures, and their labels must be formatted to be visible, 
                readable and differentiable on the printout  {\color{red}\textbf{[-5\%]}}.
        \item    Use only one font type and size for the main body of the report  {\color{red}\textbf{[-5\%]}}.
        \item    Use monospaced font (e.g. \texttt{Courier}) for computer programs, 
                functions, scripts, etc.  {\color{red}\textbf{[-5\%]}}.
        \item    Do not use monospaced font for the report body  {\color{red}\textbf{[-5\%]}}.
        \item    Glaring formatting errors, random looking formatting  {\color{red}\textbf{[-5\%]}}.
\end{itemize}

\subsection{Equations}

\begin{itemize}
        \item Number each equation in a consistent way  {\color{red}\textbf{[-5\%]}}.
        \item    Equations should be numbered to the right of the equation  {\color{red}\textbf{[-5\%]}}.
        \item    Use notation consistent with the class lectures or textbook  {\color{red}\textbf{[-5\%]}}.
        \item    Typeset equations properly (e.g. Equation Editor, LaTeX, MathType, etc.), 
do not type them as unformatted text or inject them as grainy images.  {\color{red}\textbf{[-5\%]}}.
\end{itemize}

\section{Tables and Figures}
\begin{itemize}
        \item Number and label each table and figure in a consistent way  {\color{red}\textbf{[-5\%]}}.
        \item All figures should be captioned and should be referenced in the 
                text. 
        \item    Use proper labels for plots, figures, tables – title, axis, legend, units, 
etc.  {\color{red}\textbf{[-5\%]}}.
\item   Table title should be above the table, figure title should be below the 
figure  {\color{red}\textbf{[-5\%]}}.
\item   Titles, legends, labels must be of sufficient size and quality to be 
easily readable  {\color{red}\textbf{[-5\%]}}.
\item    Make units (e.g. time) on plots/figures understandable to humans  {\color{red}\textbf{[-5\%]}}, 
for example:
\begin{itemize}
        \item   if scale exceeds 100s of sec, change to min
        \item   if scale exceeds 100s of min, change to hours
        \item   if scale exceeds 100s of hours, change to days, etc…
        \item   If solution behavior is not visible on the plot because of the 
                scale, make another plot with a different scale (or log scale) 
                that clearly shows the solution behavior  {\color{red}\textbf{[-5\%]}}.
\end{itemize}
\item    Confusing y vs. x and x vs. y  {\color{red}\textbf{[-5\%]}}.
\item    Clearly label each numerical solution inside each figure and inside each 
tables with the value of $\Delta t$ used for the numerical solution  {\color{red}\textbf{[-5\%]}}.
\item    Use sufficiently high quality figures such that they look smooth and sharp 
 {\color{red}\textbf{[-5\%]}}.
\begin{itemize}
\item    Screen shots are probably too low quality.
\item   \texttt{jpeg} and other lossy compression types are probably too low quality.
\item   High resolution and lossless vectorized image types are recommended.
        \end{itemize}
        \end{itemize}


        \section{Programming}

        \begin{itemize}
                \item Your program should be clear and readable {\color{red}\textbf{[-5\%]}}. 
                \item Include enough files for your program to run succesfully {\color{red}\textbf{[-5\%]}}. 
                \item Test your programs/functions thoroughly, after you finished testing, test 
it some more!
\item    Any programming language can be used. However, your instructor 
        recommends python for its ease of use and power.
\item   Python does not have to be used for plotting, but your instructor 
        strongly recommends python for its ease of use and power.
        \end{itemize}

        \section{Other}
  The purpose of the assignment is a comprehensive, self-contained, 
  consistently formatted report and a demonstration that you understand 
  kinetics calculations. The emphasis is not the programming itself.  If you 
  are not sure about what and how much to include in the report, imagine that 
  you have to grade it – make it concise and easy to follow. I’m being picky 
  because I want you to write good reports.  The content and formatting rules 
  are \emph{almost} universal.

\section{Presentation Content}
Present your results (ideally, include your presentation in the repository as a 
pdf). Your presentation should be :
\begin{itemize}
        \item 10 minutes (+5 minutes for questions)
        \item clear
        \item succinct
        \item well organized
\end{itemize}


\end{document}
