% use the answers clause to get answers to print; otherwise leave it out.
\documentclass[11pt,addpoints,answers]{exam}
%\documentclass[11pt,addpoints]{exam}
\RequirePackage{amssymb, amsfonts, amsmath, latexsym, verbatim, xspace, 
setspace, wasysym}
\usepackage{graphicx}

% By default LaTeX uses large margins.  This doesn't work well on exams; problems
% end up in the "middle" of the page, reducing the amount of space for students
% to work on them.
\usepackage[margin=1in]{geometry}
\usepackage{enumerate}
\usepackage[hidelinks]{hyperref}

% Here's where you edit the Class, Exam, Date, etc.
\newcommand{\class}{NPRE 560}
\newcommand{\term}{Spring 2019}
\newcommand{\assignment}{HW 3}
\newcommand{\duedate}{2019.03.14}
%\newcommand{\timelimit}{50 Minutes}

\newcommand{\nth}{n\ensuremath{^{\text{th}}} }
\newcommand{\ve}[1]{\ensuremath{\mathbf{#1}}}
\newcommand{\Macro}{\ensuremath{\Sigma}}
\newcommand{\vOmega}{\ensuremath{\hat{\Omega}}}

% For an exam, single spacing is most appropriate
\singlespacing
% \onehalfspacing
% \doublespacing

% For an exam, we generally want to turn off paragraph indentation
\parindent 0ex

%\unframedsolutions

\begin{document} 

% These commands set up the running header on the top of the exam pages
\pagestyle{head}
\firstpageheader{}{}{}
\runningheader{\class}{\assignment\ - Page \thepage\ of \numpages}{Due \duedate}
\runningheadrule

\class \hfill \term \\
\assignment \hfill Due \duedate\\
\rule[1ex]{\textwidth}{.1pt}
%\hrulefill

%%%%%%%%%%%%%%%%%%%%%%%%%%%%%%%%%%%%%%%%%%%%%%%%%%%%%%%%%%%%%%%%%%%%%%%%%%%%%%%%%%%%%
%%%%%%%%%%%%%%%%%%%%%%%%%%%%%%%%%%%%%%%%%%%%%%%%%%%%%%%%%%%%%%%%%%%%%%%%%%%%%%%%%%%%%
\begin{itemize}
        \item Show your work. 
        \item This work must be submitted online as a \texttt{.pdf} through Compass2g.
        \item Work completed with LaTeX or Jupyter earns 1 extra point. Submit 
                source file (e.g. \texttt{.tex} or \texttt{.ipynb}) along with 
                the \texttt{.pdf} file.
        \item If this work is completed with the aid of a numerical program 
                (such as Python, Wolfram Alpha, or MATLAB) all scripts and data 
                must be submitted in addition to the \texttt{.pdf}.
        \item If you work with anyone else, document what you worked on together.
\end{itemize}
\rule[1ex]{\textwidth}{.1pt}

% ---------------------------------------------
\begin{questions}
        \question (Ott Review 6.20) Describe in words, with graphs, and with 
        formulas the transient following a step change in reactivity or source:
        \begin{parts}
                \part[5] Without delayed neutrons.
                \begin{solution}
                        solution here
                \end{solution}

                \part[5] With constant delayed neutron source.
                \begin{solution}
                        solution here
                \end{solution}

                \part[5] With no approximations (no formula required).
                \begin{solution}
                        solution here
                \end{solution}

        \end{parts}

        % ---------------------------------------------
        \question (Ott Review 6.34) Estimate the time it takes to establish the 
        stable asymptotic transient for $\rho_1 < \beta$ in an initially 
        critical reactor.
                \begin{solution}
                        solution here
                \end{solution}


        % ---------------------------------------------
        \question[10] (Ott Review 6.35) Explain in terms of roots of the 
        characteristic equation:
        \begin{parts}
                \part[5] the prompt jump phenomenon
                \begin{solution}
                        solution here.
                \end{solution}
                \part[5] the delayed neutron induced transition
                \begin{solution}
                        solution here.
                \end{solution}
                \part[5] the stable period 
                \begin{solution}
                        solution here.
                \end{solution}
        \end{parts}

        
        % ---------------------------------------------
        \question[30] (Ott Problem 8.1) Find the numerical value of $p^{00}$, 
        the flux after a prompt jump for which the increase due to delayed 
        neutrons is just compensated by Doppler feedback, for an LWR from the 
        typical $\lambda$ and $\gamma/\beta$ values given in the text. Discuss 
        why $p^{00}$ may vary between reactors (e.g. the SEFOR reactor 
        discussed in the text).

        \begin{solution}
                solution here.
        \end{solution}

        % ---------------------------------------------
        \question[15] (Ott Review 8.1) Define each term, give an example of the 
        physical phenomena involved, and an example of a transient for each: 
        \begin{parts}
                \part[5] Energy coefficient.
                \part[5] Temperature coefficient. of reactvity.
                \part[5] Power coeffciient.
        \end{parts}
        \begin{solution}
                solution here.
        \end{solution}
       
       
\end{questions}



%\bibliographystyle{plain}
%\bibliography{hw01}
\end{document}
