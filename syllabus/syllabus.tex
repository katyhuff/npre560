\documentclass[11pt]{article}
\usepackage[inner=1in,outer=1in,top=1in,bottom=1in]{geometry}
\pagestyle{empty}
\usepackage{placeins}
\usepackage{graphicx}
\usepackage{fancyhdr, lastpage, bbding, pmboxdraw}
\usepackage[usenames,dvipsnames]{color}
\definecolor{darkblue}{rgb}{0,0,.6}
\definecolor{darkred}{rgb}{.7,0,0}
\definecolor{darkgreen}{rgb}{0,.6,0}
\definecolor{red}{rgb}{.98,0,0}
\usepackage[colorlinks,pagebackref,pdfusetitle,urlcolor=darkblue,citecolor=darkblue,linkcolor=darkred,bookmarksnumbered,plainpages=false]{hyperref}
%\renewcommand{\thefootnote}{\fnsymbol{footnote}}

\pagestyle{fancyplain}
\fancyhf{}
\lhead{ \fancyplain{}{\CourseTitle} }
%\chead{ \fancyplain{}{} }
\rhead{ \fancyplain{}{\CourseSemester \CourseYear} }
%\rfoot{\fancyplain{}{page \thepage\ of \pageref{LastPage}}}
\fancyfoot[RO, LE] {page \thepage\ of \pageref{LastPage} }
\thispagestyle{plain}
\usepackage{tabularx}
\usepackage{amsmath}

%%%%%%%%%%%%%%%%%%%%%%%%%%%%%%%%%%%%
\usepackage{xspace}
\newcommand{\CourseNumber}{NPRE560}
\newcommand{\CourseTitle}{Reactor Kinetics and Dynamics\xspace}%
\newcommand{\CourseSemester}{Fall\xspace}%
\newcommand{\CourseYear}{2024\xspace}%
\newcommand{\CourseDays}{MW\xspace}%
\newcommand{\CourseStart}{1:00pm\xspace}%
\newcommand{\CourseEnd}{2:50pm\xspace}%
\newcommand{\CourseInstructor}{Prof. Kathryn Huff}
\newcommand{\CourseInstructorEmail}{kdhuff@illinois.edu}
\newcommand{\CourseRoom}{305\xspace}%
\newcommand{\CourseBuilding}{Materials Science \& Engineering Building\xspace}%
\newcommand{\CourseUniversity}{University of Illinois, Urbana-Champaign\xspace}%
\newcommand{\TeachingAssistant}{TA Name\xspace}%
\newcommand{\TAOfficeHourDays}{Wednesdays\xspace}%
\newcommand{\TAOfficeHourStart}{1:00pm\xspace}%
\newcommand{\TAOfficeHourEnd}{3:00pm\xspace}%
\newcommand{\TAOfficeHourPlace}{123 Talbot Laboratory\xspace}
%\newcommand{\Course<++>}{<++>}
%\newcommand{\Course<++>}{<++>}
%%%%%%%%%%%%%%%%%%%%%%%%%%%%%%%%%%%%
\title{\CourseNumber: \CourseTitle\\}
\author{\CourseUniversity}
\date{\CourseSemester \CourseYear}
\begin{document}
\maketitle
%\setlength{\unitlength}{1in}
\renewcommand{\arraystretch}{2}
\begin{center}
\begin{table}[h]
\begin{tabularx}{\textwidth}{rXrX}
\hline
\textbf{Instructor:} & \CourseInstructor & \textbf{Time:} & \CourseDays \CourseStart -- \CourseEnd \\
\textbf{Email:} &  \href{mailto:\CourseInstructorEmail}{\CourseInstructorEmail} & \textbf{Place:} & \CourseRoom \CourseBuilding\\
\hline
\end{tabularx}

\end{table}
\end{center}

\paragraph{Course Pages:}
\begin{enumerate}
        \item \url{https://canvas.illinois.edu}
        \item \url{https://github.com/katyhuff/\CourseNumber}
        \item \url{https://classroom.github.com/NPRE5XX/\CourseNumber}
\end{enumerate}

%\paragraph{TA Office Hours:} The teaching assistant for the course,
%\TeachingAssistant, will hold office hours \TAOfficeHourDays from
%\TAOfficeHourStart to \TAOfficeHourEnd in \TAOfficeHourPlace.

\paragraph{Office Hours:} Prof. Huff will hold office hours by appointment 
only.  If possible, please plan to meet in her office, 118 Talbot Laboratory at 
104 S. Wright St. If an in-person meeting is impossible, please request a Zoom 
meeting.  Please make an appointment at \url{katyhuff.youcanbook.me}.  If your 
colleagues might be helpful, please discuss your questions with them directly 
before scheduling office hours.

\paragraph{Main References:}
A few essential references for this course will be assigned as readings and 
will be the source of homework problems. The key texts for this course are
\cite{ott_introductory_1985} and
\cite{hetrick_dynamics_1993}.
I also recommend 
\cite{stacey_nuclear_2007}, \cite{bell_nuclear_1970} and
\cite{duderstadt_transport_1979} for conceptual review as well as
\cite{lewis_computational_1993} for computational methods details.
\bibliographystyle{unsrt}
\renewcommand{\refname}{\normalfont\selectfont\normalsize}
\bibliography{syllabus}

\paragraph{Objectives:}
\begin{itemize}
\item Incorporate delayed neutrons into diffusion and transport neutron balances
\item Develop the point reactor kinetics equations
\item Apply, analytically and numerically, the point reactor kinetics equations
\item Apply space dependent, multigroup reactor kinetics
\item Understand and interperet reactivity measurements
\item Analyze dynamics via reactor noise
\item Interrogate advanced topics 
\item Engage critically with original kinetics and dynamics research
\end{itemize}

\paragraph{Prerequisites:}
\begin{itemize}
        \item NPRE455 or equivalent (required) 
        \item NPRE555 (desired, but optional)
\end{itemize}

\paragraph{Grading Policy:} Grades will be assigned as a weighted sum of the
following work. Approximately 5 homework assignments, 3 computational projects, 
and occasional engagement with the literature (journal club activity) will comprise the 
grade.

\paragraph{Homework:} There will be a small number of homework assignments 
throughout the semester, approximately one every three weeks.

\paragraph{Computational Projects:} The computational projects will be 
submitted via GitHub Classroom and will be spaced evenly through the semester. 
The final project will be presented during the time reserved for this course's 
final exam.

\paragraph{Journal Club:} A small library of original research papers will be 
provided for students to engage with. All students will be asked to submit 
short reviews of the papers. Individual students will each be assigned to 
present a review and summary of at least one of these papers to the class (15 
minutes) over the course of the semester.  Participation in class discussion 
during this activity by the student audience will constitute a fraction of the 
awarded grade for this activity.

\begin{table}[h]
\begin{tabularx}{\textwidth}{Xr}
        \textbf{Work} & \textbf{Weight} \\
\hline
\textbf{Journal Club}    & (20\%)  \\
\textbf{Homework}    & (20\%)  \\
\textbf{Project 1}    & (20\%)  \\
\textbf{Project 2}    & (20\%)  \\
\textbf{Final Project}  & (20\%)  \\
\hline
\textbf{Total}       & (100\%)\\
\end{tabularx}
\end{table}

\paragraph{Important Dates:}
\begin{center} \begin{minipage}{3.8in}
\begin{flushleft}
%Midterm \#1      \dotfill 10:00-10:50pm, 3, 2017 \\
%Midterm \#2      \dotfill 10:00-10:50pm, 7, 2017\\
%Project Deadline \dotfill ~Month Day \\
Project 1    \dotfill  Feb 26, 2019\\
Project 2  \dotfill  Apr 2, 2019\\
Final Project    \dotfill 7:00-10:00pm, May 3, 2019\\
\end{flushleft}
\end{minipage}
\end{center}

\paragraph{Class Policies:}


\begin{itemize}
\item[] \textbf{Integrity:} This is an institution of higher
learning. You will be swiftly ejected from the course if you are caught
undermining its integrity. Note the
\href{http://www.provost.illinois.edu/academicintegrity/students.html}{Student's
Quick Reference Guide to Academic Integrity} and the
\href{http://studentcode.illinois.edu/article1_part4_1-401.html}{Academic
Integrity Policy and Procedure}.
\item[] \textbf{Attendance:} Regular attendance is mandatory. Request approval for absence for extenuating circumstances prior to absence.
\item[] \textbf{Electronics:} Active participation is essential and expected.
        Accordingly, students must turn off all electronic devices (laptop,
        tablets, cellphones, etc.) during class. Exceptions may be granted for
        laptops if engaging in computational exercises or taking notes.
\item[] \textbf{Collaboration:} Collaboratively reviewing course materials and studying for exams with fellow students can be enriching.  This is recommended.  However, unless otherwise instructed, homework assignments are to be completed independently and materials submitted as homework should be the result of one's own independent work.
\item[] \textbf{Late Work:} Late work will not be accepted. Plan ahead.

\item[] \textbf{Make-up Work:} There will be no negotiation about late work
        except in the case of absence documented by an absence letter from the
        Dean of Students.  The university policy for requesting such a letter
        is in
        \href{http://studentcode.illinois.edu/article1_part5_1-501.html}{the
        Student Code}. Please note that such a letter is appropriate for many
        types of conflicts, but that religious conflicts require special early
        handling. In accordance with university policy, students seeking an
        excused absence for religious reasons should complete the
	\href{http://odos.illinois.edu/community-of-care/resources/students/religious-observances/}{Request for Accommodation for Religious Observances Form}
        The student should submit this
        form to the instructor and the Office of the Dean of Students by the end of the
        second week of the course to which it applies.


\item[] \textbf{Grade Disputes:} It is important that you understand and agree
        with the grade you receive on assignments and exams. If you would like
        to dispute your score, you must send an explanation by email to Prof.
        Huff within one week of recieving the grade.
        \textbf{Do not expect me to regrade anything while in conversation with
        you} as that would not be fair to the other students in the class, whose
        homeworks were graded without them present.  If you request a regrade,
        be aware that the entire assignment will be regraded and is subject to
        double-jeopardy: it is possible that your score will go down.
        Regrade requests should be based on an error on my part (e.g., adding
        up the points incorrectly) or what you suspect is a misunderstanding of
        your work (e.g., arriving at the correct answer using an unexpected
        technique). Regrade requests that argue with the rubric (e.g., ``this is
        wrong, but you took too many points off'') will be returned without
        consideration.
        \textbf{Your work should stand alone.} If an assignment is disorganized or
        ambiguous, and requires an extensive explanation to the grader, you
        will likely still lose points. The homeworks not only evaluate your
        understanding of the material - they also evaluate your ability to
        communicate that understanding clearly and concisely.
\end{itemize}

\paragraph{Accessibility:} I hope that this course will be inclusive and
accommodating for all learners. As such, I am committed upholding the vision
and values of \href{http://www.inclusiveillinois.illinois.edu/index.html}{Inclusive Illinois}
in my
classroom.  With regard to accommodating all learners, please note that many
resources are provided through
\href{http://disability.illinois.edu/academic-support/accommodations}{the
Division of Disability Resources and Educational Services}.  To request
particular accommodations, please contact me as soon as possible so that we can
work out any necessary arrangements.

\paragraph{Safety:}
Emergencies can happen anywhere and at any time, so it’s important that we take
a minute to prepare for a situation in which our safety could depend on our
ability to react quickly. Take a moment to learn the different ways to leave
this building. If there’s ever a fire alarm or something like that, you’ll know
how to get out and you’ll be able to help others get out. Next, figure out the
best place to go in case of severe weather - we’ll need to go to a low-level in
the middle of the building, away from windows. And finally, if there’s ever
someone trying to hurt us, our best option is to run out of the building. If we
cannot do that safely, we’ll want to hide somewhere we can’t be seen, and we’ll
have to lock or barricade the door if possible and be as quiet as we can. We
will not leave that safe area until we get an Illini-Alert confirming that it’s
safe to do so. If we can’t run or hide, we’ll fight back with whatever we can
get our hands on. If you want to better prepare yourself for any of these
situations, visit \url{police.illinois.edu/safe}. Remember you can sign up for
emergency text messages at \url{emergency.illinois.edu}. This
\href{http://police.illinois.edu/dpsapp/wp-content/uploads/2017/08/syllabus-attachment.pdf}{one-page-handout}
discusses the Illinois Run-Hide-Fight strategy.


\paragraph{Other Resources:}
University students typically experience a wide range of stressors during their
time on campus. Accordingly, campus resources exist to help students manage
stress levels, mental health, physical health, and emergencies while navigating
this environment. I hope you will take advantage of these campus resources as
soon as they can be of help.

\begin{itemize}
\item \href{https://campusrec.illinois.edu/}{The Campus Recreational Centers}
\item \href{http://counselingcenter.illinois.edu/}{The Counselling Center}
\item \href{http://www.mckinley.illinois.edu/clinics/mental\_health.htm}{The McKinley Mental Health Clinic}
\item \href{http://odos.illinois.edu/emergency/}{The Emergency Dean}
\end{itemize}

\pagebreak
\FloatBarrier
\renewcommand{\arraystretch}{1}
\begin{table}[h]
\begin{center}
\begin{tabular}{lllcllll}
\multicolumn{8}{c}{\textbf{Course Schedule:}\textit{ Note that this schedule is subject to change}}\\
&&&&&&&\\
\textbf{Date} & \textbf{Week} & \textbf{Day} & \textbf{Unit} & \textbf{Chap.} & \textbf{Chap.} & \textbf{HW} & \textbf{HW}\\
              &  &  & & \textbf{Ott}& \textbf{Hetrick} & \textbf{Given} & \textbf{Due}\\
\hline
\hline
01-15 & 1 & T  & \textbf{No Class}               &    & 1 &      & \\
01-17 & 1 & Th & Delayed Neutrons       & 1\&2  & 1 &      & \\
01-22 & 2 & T  & Simple PKE             & 3  & 1 &      & \\
01-24 & 2 & Th & Perturbation           & 4  & 1 & CP1  & \\
01-29 & 3 & T  & Exact PKE              & 4  & 1 & HW1  & \\
01-31 & 3 & Th & Inhour                 & 5  & 1 &      & \\
02-05 & 4 & T  & Basic Stability        & 5  & 2 &      & \\
02-07 & 4 & Th & \textbf{No Class}               &    &  & HW2  & HW1\\
02-12 & 5 & T  & PKE Solutions          & 6  & 2 &      & \\
02-14 & 5 & Th & Microkinetics          & 7  & 2 &      & \\
02-19 & 6 & T  & Prompt Jump            & 8  & 5 & HW3  & HW2\\
02-21 & 6 & Th & Ramp/Step              & 8  & 5 &      & \\
02-26 & 7 & T  & Reactivity Measurement & 9  & 2 & CP2  & CP1\\
02-28 & 7 & Th & Feedbacks              & 10 & 3 &      & \\
03-05 & 8 & T  & Feedbacks              & 10 & 3 &      & \\
03-07 & 8 & Th & Feedbacks              & 10 & 3 &      & \\
03-12 & 9 & T  & Numerical Methods      & 10 & 4 &      & \\
03-14 & 9 & Th & Numerical Methods      & 10 & 4 & HW4  & HW3\\
03-19 & 10 & T  & \textbf{Spring Break} &    &   &      & \\
03-21 & 10 & Th & \textbf{Spring Break} &    &   &      & \\
03-26 & 11 & T  & Linear Stability      & 10 & 6 &      & \\
03-28 & 11 & Th & Linear Stability      & 10 & 6 &      & \\
04-02 & 12 & T  & Non-linear Stability  & 10 & 7 & CP3  & CP2\\
04-04 & 12 & Th & Reactor Applications  & 10 & 7 &      & \\
04-09 & 13 & T  & Reactor Applications  & 10 & 7 &      & \\
04-11 & 13 & Th & Space Dependence      & 11 & 8 &      & \\
04-16 & 14 & T  & Space Dependence      & 11 & 8 & HW5  & HW4\\
04-18 & 14 & Th & Space Dependence      & 11 & 8 &      & \\
04-23 & 15 & T  & Space Dependence      & 11 & 8 &      & \\
04-25 & 15 & Th & Space Dependence      & 11 & 8 &      & \\
04-30 & 16 & T  & Open Questions        &    &   &      & HW5\\
05-02 & 16 & Th & \textbf{Reading Day}  &    &   &      & \\
05-03 & 16 & M & \textbf{Final Project} &    &   &      & CP3 \\
\end{tabular}
\end{center}
\end{table}
%%%%%% THE END 
\end{document} 
